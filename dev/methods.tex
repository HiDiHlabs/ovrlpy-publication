\documentclass{article}

\title{ovrl.py, a Python QC module discovering overlapping signal in spot-based data.}

\author{Sebastian Tiesmeyer, Naveed Ishaque, Roland Eils}

\begin{document}

\section{Methods}

This package is a lightweight Python module for QC of spot-based data. It is designed to be used in an analysis pipeline, where it can be used to evaluate the performance of analysis algorithms and their coping strategies for overlapping signals. The package provides a main analysis function and a number of plotting and visualization tools. 

\subsection{Recovering overlapping signals}

Overlapping signals are a common problem in spot-based data. The package provides a function to recover overlapping signals by splitting the data in half along the vertival (z-axis) and comparing the expression vectors in the two halves. A symmetrized  Kullback-Leibler divergence is used to measure the similarity between the two vectors at each location in the horizontal plane. Ovrlpy finds peaks in the resulting divergence map and reports these as regions that potatially contain overlapping signal.



\end{document}